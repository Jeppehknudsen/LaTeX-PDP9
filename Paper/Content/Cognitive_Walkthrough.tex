\chapter{The Cognitive Walkthrough}
\label{Cognitive Walkthrough}

\begin{itemize}
	\item Formål efterfulgt af psykologisk baggrund
	\begin{itemize}
		\item Motivationen fra TC's side?
		\item Lavpraktisk og på ingen måde emnetungt at undersøge
		\item Kræver ikke nødvendigvis musikere som forsøgspersoner
		\item Man vil hellere bare begynde at bruge et apparat i stedet for at læse en manual
		\item Hvad er den psykologiske baggrund for problemløsning med nye produkter?
		\item Human Factors Egineering?
	\end{itemize}

	
	\item Byg bro til at snakke om konceptuelle modeller (Research question)
	\begin{itemize}
		\item Er det tydeligt nok, at drejeknapperne også kan trykkes ned?
		\item Forstår man rækkefølgen for valg af effekter?
	\end{itemize}
	
	\item Hvad er en cognitive walkthrough?
	\begin{itemize}
		\item En metode til undersøgelse af learnability
		\item Passer perfekt til "hellere bare prøve sig frem"
		\item Cost effective og nem at udføre
		\item Kan udføres uden at have egentlig brugere med
		\item Kilder siger, man typisk bruger usability experts og developers
		\item Denne test udføres dog med novices
		\begin{itemize}
			\item Ønske fra TC's side, fordi det er nemt at skaffe deltagere
			\item Forskellen på novices og experts i usability testing
			\item Jeg ender derfor med at bruge studerende i Aalborg
		\end{itemize}
	\end{itemize}
	
	
	\item Fremgangsmåde for Cognitive Walkthrough
	\begin{itemize}
		\item Først defineres target users
		\item herefter defineres den konceptuelle model for fokus (at tilskrive effekt til plads)
		\item Den konceptuelle model splittes herefter ud i de interaktioner, den består af
		\item Action Sequence (interaktionstrinene)
		\item Overvejelser om, drej og tryk skal være én eller to actions
		\item De fire spørgsmål, der skal stilles til hvert trin i action sequence
		\begin{itemize}
			\item Definer, hvad hvert spørgsmål fokuserer på: visibility, signifiers, feedback
			\item Hvordan vil jeg besvare disse spørgsmål? Observation, Dialog, video osv.
		\end{itemize}
	\end{itemize}
	
	
	\item Kategorisering og identificering af fejl
	\begin{itemize}
		\item I stedet for bare at finde fejl, bruges dette til at se på deres grad af vigtighed
		\item Foreslået af Jesper - Materiale fra Jan Stage (få styr på kildebetegnelse)
		\item Det skal tydeligt defineres, hvad der menes med kosmetisk, alvorlig, og kritisk
		\item Hvordan jeg implementerer det i testen (0, 1, 2, 3)
	\end{itemize}
	
	
	\item Selve udførelsen af testen
	\begin{itemize}
		\item Hvor findes der deltagere, og hvem er de
		\item Rækkefølgen i en session
		\item videooptagelse for en sikkerhedsskyld af deres hænder
	\end{itemize}
	
	
	\item Databehandlingen?
	\begin{itemize}
		\item Smart at anvende en metode, hvor databehandling er en del af opskriften
		\item Kategoriseringen af fejl - Der tages gennemsnit osv.
	\end{itemize}
	
	
	\item Resultater
	\item diskussion/konklusion
	\item Validation test - Coming soon!
	\begin{itemize}
		\item Måske ikke lige det re-design, jeg havde forestillet mig
		\item Jeg er i et design team, hvor vi er et sted i en tidsprocess, hvor det måske ikke er muligt at foretage store ændringer
		\item Mit resultat fortæller ikke, hvordan det skal se ud, men hvordan det IKKE skal se ud
		\item Jeg skal undersøge, om deres nye designforslag fungerer
		\item Hypotese: Der kan vise sig at være forskel fra novicer til potentielle eksperter
	\end{itemize}
\end{itemize}

\begin{itemize}
	\item 
\end{itemize}














