\chapter{Create/Delete Functionality User Study}
\label{Create_Delete_Test}
%Overvej hvad afsnittet kunne kaldes: "Board Management test" eller lignende. "Create/Delete både lyder dumt og ser latterligt ud.

\begin{itemize}
	\item Formål
	\begin{itemize}
		\item TC's ønske samt egen motivation
		\item På sigt ønskes det integreret i toneprint app
		\item Research question
	\end{itemize} 


	\item Fra "Formål" til "Cases"
	\begin{itemize}
		\item Nuværende designforslag i forhold til Create/Delete
		\item analyse af dette udfra teori om psykologi og interaction design
%It's unclear whether they will know to hold the switch in the right mode to activate create/delete board. - Affordance/signifiers/visibility
%Jeg kan sige nok så meget ud fra interaktionsdesign osv. I sidste ende vil det nok være federe, hvis jeg kunne få nogle forsøgspersoner til at understøtte det (Motivation?)
		\item Revurderede designforslag med hjælp fra Jess
		\item Forskellen på de to cases (Beskriv dem hver især og sammenlign så)
		\item Test af disse to med fokus på, at finde usability problemstillinger
	\end{itemize}
	
	\item Low-fidelity test med en fysisk kasse og papirstykker ovenpå
	\begin{itemize}
		\item Lo-fi fordi kun den ene case eksisterer som funktionel prototype
		\item Kontrol variablerne skal holdes konstant
		\item Dog har den LEDer med lys i for at få den lidt tættere på virkeligheden
		\item Rops anbefaling at gøre det på denne måde, fordi Jeg kan ikke papirteste fysiske knapper (Her skal der inddrages teori om Lo-Fi tests)
	\end{itemize}
	
	
	\item Test design
	\begin{itemize}
		\item Målet er at belyse gode og dårlige usability aspekter for dem begge
		\item Det foregår kvalitativt, da jeg gerne vil have dem selv til at sætte ord på det
		\item Within Subject Design - Hvorfor?
		\item Retrospective Think aloud (RTA) - Optagelserne gennemgåes med dem selv
		\item Video- og lydoptagelse af hver session
		\item Rop foreslår at blande interviews med spørgeskemaformatet.
		\item Forsøgspersoner - Lo-Fi => De skal være trænet i brugen af produkttypen
		\item De må ikke kende produktet i forvejen (Bias, duh)
		\item For lethedens skyld anvendes folk fra firmaet
		\item Samtykkeerklæring samt fremgangsmåde
	\end{itemize}


	\item Databehandling
	\begin{itemize}
		\item Motivationen - Hvad slags data vil jeg gerne have ud af testen
		\item Mind læseren om, at det foregår kvalitativt => En bestemt slags data
		\item En slags Card Sorting?
		\item Først noteres præcist hvad de gør - Hver eneste handling efter hinanden
		\item Herefter kategoriseres de, så dem der er ens også er formuleret ens
		\item Som en ekstra runde kigges der også på rækkefølgen for deres gæt
	\end{itemize}

	\item Resultater
	\item Diskussion
	\item Validation test - Coming soon!
	\begin{itemize}
		\item 
		\item Intentionen er at udføre det magen til første test
		\item Fra papirprototype til fuldt funktionel prototype
		\item Ændring i manuskriptet
		\item 10 nye forsøgspersoner
		\item Ændring i opgaverne
		\begin{itemize}
			\item Jeg kan ikke bede flere forsøgspersoner om at slette samme board
			\item Boards er nummereret og kan ikke komme igen, når de er slettet
			\item De bedes i stedet slette det ældste board i rækken
			\item På den måde, bliver deres forståelse af rækkefølgen også undersøgt
		\end{itemize}
	\end{itemize}
\end{itemize}



















